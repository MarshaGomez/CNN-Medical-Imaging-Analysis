\documentclass{book}
\usepackage{hyperref}
\usepackage{geometry}
\usepackage{graphicx}
\usepackage{apacite}
\usepackage{natbib}
\usepackage{float}
\usepackage{subfigure}
\usepackage{array,booktabs} % Table packages

\hypersetup{
    colorlinks=true,
    linkcolor=black,
    filecolor=magenta,      
    urlcolor=cyan,
    pdftitle={Sharelatex Example},
    bookmarks=true,
    pdfpagemode=FullScreen,
}

\pagestyle{empty}
\graphicspath{{img/}}
\begin{document}
    \begin{titlepage}
        \centering
        \includegraphics[width=6cm]{logo-unipi.png}
        \vfill
        \vspace{1.5cm}
        {\huge\textsc{Computational Intelligence and Deep Learning}\par}
        {\Large
            Educational Version\\
            2020\\
            \vspace{2cm}
            A. Schiavo - M. Gómez - M. Daole \\
            \vspace{2cm}
            \today
        }    
        \vfill
        \vfill
    \end{titlepage}
    \tableofcontents


    % --------------------------------------------------------------------
    \chapter{Introduction}

    Deep learning rose to its prominent position in computer vision when neural networks started outperforming other methods on several high-profile image analysis. Healthcare providers generate and capture enormous amounts of data containing extremely valuable signals and information, at a pace far surpassing what \textit{traditional} methods of analysis can process. Machine learning therefore quickly enters the picture, as it is one of the best ways to integrate, analyze and make predictions based on large, heterogeneous data sets. 


    % --------------------------------------------------------------------

    \chapter{Convolutional Neural Network for Medical Imaging Analysis}

    On this investigation, the main objective is to perform abnormality classification in mammography using Convolutional Neural Networks for Medical Imaging Analysis. This laboratory research will be development with a standart evaluation data set in the area of decision support systems in mammography, the \textit{Digital Curated Breast Imaging Subset of Database for Screening Mammography} \textbf{(CBIS DDSM)}
    
    \section{Original Dataset}
        The dataset we will focus on is an updated and standardized version of the Digital Database for Screening Mammography (DDSM). The DDSM is a database of 2,620 scanned film mammography studies. It contains normal, benign, and malignant cases with verified pathology information. Few well-curated public datasets have been provided for the mammography community. These include the DDSM, the Mammographic Imaging Analysis Society (MIAS) database, and the Image Retrieval in Medical Applications (IRMA) project. Although these public data sets are useful, they are limited in terms of data set size and accessibility.

        \begin{center}
            \begin{minipage}{0.9\linewidth}
                \vspace{5pt}%margen superior de minipage
                {\small
                    Lee, Rebecca Sawyer, et al. \textit{'A curated mammography data set for use in computer-aided detection and diagnosis research'} Scientific data 4 (2017): 170177.
                }
                \vspace{5pt}%margen inferior de la minipage
            \end{minipage}
        \end{center}

        The images have been decompressed and converted to DICOM format. Updated ROI segmentation and bounding boxes, and pathologic diagnosis for training data are also included. The data set contains 753 calcification cases and 891 mass cases, providing a data-set size capable of analyzing decision support systems in mammography.

        
        \begin{table}
            \begin{tabular}{m{5cm}   m{8cm} }
                \midrule
                \textbf{Design Types} & -Design Types and Parallel group design.  -Feature extraction objective.  -Image processing objective \\
                \midrule
                \textbf{Measurement Type} & Mammography \\ 
                \midrule
                \textbf{Technology Type} & Digital curation \\ 
                \midrule
                \textbf{Factor Type} & Diagnosis \\
                \midrule
                \textbf{Sample Characteristic} & Homo sapiens\\
                \midrule
            \end{tabular}
            \caption{\label{tab:DDSM}DDSM Data Set description.}
        \end{table}

        Considering the benefits of using deep learning in image classification problem (e.g., automatic feature extraction from raw data), develop a deep Convolutional Neural Network (CNN) that will be trained to read mammography images and classify them into the following five instances:

        \begin{itemize}
            \item Normal
            \item Benign Calcification
            \item Benign Mass
            \item Malignant Calcification
            \item Malignant Mass
        \end{itemize}

        A simple example of the image provided from the original dataset:

        \begin{figure}[h]
            \subfigure[Mass, Benign]{\includegraphics[width=3cm]{intro-masses-benign.png}}
            \hfill
            \subfigure[Mass, Malignant]{\includegraphics[width=3cm]{intro-masses-malign.png}}
            \hfill
            \subfigure[Calcification, Benign]{\includegraphics[width=3cm]{intro-calc-benign.png}}
            \hfill
            \subfigure[Calcification, Malignant]{\includegraphics[width=3cm]{intro-calc-malign.png}}
            \hfill
            \caption{Medical Image Representation}
        \end{figure}
         
        \section{Data Records}
        The images are distributed at the full mammography and abnormality level as DICOM files. Full mammography images include both MLO and CC views of the mammograms. Abnormalities are represented as binary mask images of the same size as their associated
        mammograms. 

        \begin{figure}[h]
            \subfigure[Classes: Baseline patch(0), Mass, benign(1), Mass, malignant(2), Calcification, benign(3), Calcification, malignant(4)]{\includegraphics[width=7cm]{intro-dataset-class.png}}
            \hfill
            \subfigure[Pathology: Benign(0), Malignant(1)]{\includegraphics[width=7cm]{intro-dataset-calc-mass.png}}
            \hfill
            \subfigure[Pathology: Benign(0), Malignant(1)]{\includegraphics[width=7cm]{intro-dataset-calc.png}}
            \hfill
            \subfigure[Pathology: Benign(0), Malignant(1)]{\includegraphics[width=7cm]{intro-dataset-masses.png}}
            \hfill
            \caption{Data Representation}
        \end{figure}

    
    As we mention previously,  \textit{CBIS DDSM:} Curated Breast Imaging Subset of Digital Database for Screening Mammography.

    A description of the dataset is provided in:

    % --------------------------------------------------------------------

    \chapter{CNN from Scratch with Python}

    % --------------------------------------------------------------------
    
    \chapter{Pre-Trained CNN}

    % --------------------------------------------------------------------

    \chapter{Baseline Abnormality detection in mammography}

    % --------------------------------------------------------------------

    \chapter{Ensemble of Neural Networks}
    \section{Examples of Predictions}

    % --------------------------------------------------------------------

\end{document}